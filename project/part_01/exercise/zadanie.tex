\date{\today}
\documentclass[12pt]{article}
\usepackage{graphicx}
\usepackage[T1]{fontenc}
\begin{document}
\section{Zadania (Tasks)}
\subsection{Podstawowe}
\paragraph{Wprowadzenie obiektu}
\paragraph{Przeniesienie obiektu z wejścia przez serwo do miejsca analizy}
\paragraph{Przeprowadzenie Analizy. Wyznaczenie częstotliowści R,G,B}
\paragraph{Określenie koloru na podsatwie częstotliwości}
\paragraph{Wyznaczenie odpowiedniego kątu}
\paragraph{Ustalenie kontenera}
\paragraph{Przesunięcie zsuwu przez serwo o wyznaczony kąt}
\paragraph{Zepchnięcie, zwolnienie obiektu przez serwo}
\paragraph{Reset serwa przesuwającego obiekt do pozycji startowej}
\subsection{Określenia czasowe}
\subsubsection{Względnie( ograniczenia następstw)}
Proces powtarzamy tyle razy ile mamy obiektów.\\
\begin{enumerate}
\item Wprowadzony zostaje obiekt,ale  nie wcześniej niż serwo 1 jest na pozycji "zerowej".
\item Po wprowadzeniu obiektu Serwo 1 przesuwa obiekt pod sensor.
\item Obiekt jest badany pod względem natężeń RGB.
\item Po przebadaniu Serwo 2 przesuwa zsuw pod odpowiednim kątem.
\item Serwo nr 1, delikatnie zazebiając się w czasie z przesunięciem zsuwu przez serwo 2, może zacząć wypychać obiekt spod sensora na zsuw.
\item Cofnięcie serwa nr 1 do pozycji "zerowej" po ustawieniu zsuwu pod odpowiednim kątem.
\end{enumerate}
\subsubsection{Wartości( przybliżone)}
W kontekście czasowym najbardziej optymalne jest wprowadzanie kulek na wejście w równych odstępach czasowych, tj bezpośrednio po resecie serwa górnego(nr 1) do pozycji startowej, chociaż jeśli założymy, że musimy posortować cały zbiór i tak, a czas nie gra roli to jest to sprawa drugorzędna. Podanie obiektu przez serwo do miejsca analizy powinno zająć około 600 ms. Częstotliwości R,G,B powinnny zostać wyznaczone w czasie 50 ms(x3 = 150ms). Określenie koloru około 1000 ms.  Przesunięcie zsuwu 700-800 ms. Przepchnięcie obiektu nad zsuw 300 ms.
powrót do pozycji startowej 700 ms. Rozpoczęcie manewru zepchnięcia kuli i ustawienia zsuwu mogłoby się delikatnie zazębiać w celu zoptymalizowania urządzenia.
\end{document}