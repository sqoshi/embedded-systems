\documentclass[12pt]{article}

\usepackage[T1]{fontenc}
\usepackage{listings}
\usepackage{color}

\definecolor{dkgreen}{rgb}{0,0.6,0}
\definecolor{gray}{rgb}{0.5,0.5,0.5}
\definecolor{mauve}{rgb}{0.58,0,0.82}

\lstset{frame=tb,
  language=C++,
  aboveskip=3mm,
  belowskip=3mm,
  showstringspaces=false,
  columns=flexible,
  basicstyle={\small\ttfamily},
  numbers=none,
  numberstyle=\tiny\color{gray},
  keywordstyle=\color{blue},
  commentstyle=\color{dkgreen},
  stringstyle=\color{mauve},
  breaklines=true,
  breakatwhitespace=true,
  tabsize=3
}

\begin{document}
\section{Polecenia}
Załóżmy, że autko "dorobiło się" możliwości patrzenia. Ma więc teraz całkiem sporo możliwości poruszania się.  Pora to wykorzystać. Wykonaj jedno z poniższych zadań (niektóre są łatwiejsze, inne - ciekawsze). Budując algorytm realizujący któreś z nich, polegaj na dostępnych systemach: zliczaniu obrotów kół i radarze sonicznym. Jak zwykle, pisz w pseudokodzie z sensownie opisanymi metodami tak, żeby było wiadomo, co która robi.
\subsubsection{}Zaimplementuj mechanizm płynnego omijania przeszkody. Autko, napotkawszy na swojej drodze przeszkodę (np. plecak), powinno spróbować ją ominąć bez zatrzymania się, tj. nie wykonując operacji "stop-skręć w miejscu-jedź". Po ominięciu przeszkody pojazd powinien kontynuować jazdę w mniej więcej tym samym kierunku, jak przed rozpoczęciem omijania. Wskazówka: pomyśl, jak osoba niewidoma omijałaby przeszkodę korzystając z laski. Możesz w ten sposób wykorzystać obracający się na serwomotorze sonar jako "próbnik".
\subsubsection{}Przypomnij sobie, jak wygląda korytarz w sali 317. "Wyprowadź" autko na samodzielny spacer: "ustaw" je przed wyjściem z sali 317.2 w kierunku korytarza. Auto powinno po wyjechaniu z sali skręcić w lewo i jechać w kierunku serwerowni. Po dotarciu do końca korytarza auto powinno skręcając w prawo podjechać do ściany z oknami i ponownie skręcić w prawo. Następnie powinno jechać tak długo, aby móc skręcić i wrócić do sali 317.2. Po przejechaniu progu powinno się zatrzymać. (UWAGA! Autko nie powinno pojechać za daleko w kierunku wyjścia z na klatkę schodową!)
\subsubsection{}Zaprojektuj model sprężyny, wykorzystując prawo Hooke'a. Model działania możesz obejrzeć tutaj. Auto jest masą. Zamiast rozciągnięcia/ściśnięcia sprężyny użyj odległości autka od przeszkody. Sprężyna jest "twarda" - nie oscyluje.\\
Działanie, którego na razie nie sprawdzisz (ale pewnie kiedyś się uda) powinno być takie: autko ma podjechać do ruchomej przeszkody i powoli się przed nią zatrzymać w odległości ok 100 cm (moment zerowej energii). Zbliżając się, ma kontrolować względny ruch przeszkody i odpowiednio przyspieszać (gonić ją) jeśli przeszkoda się oddala, lub zwalniać (zawracać!), jeśli się przybliża. Po zatrzymaniu się przed przeszkodą, ma utrzymywać od niej stałą odległość tak, aby zawsze znajdować się w punkcie "minimum energii".

\section{Rozwiązanie}
\begin{lstlisting}
void z2{
	%Deklaracja zmiennych
	distance=0;
	wheelCounter=0;
	turnLeftDegrees(90);
	%zliczanie obrotow
	attachPCINT(digitalPinTOPCINT(INTINPUT0),increment,RISING);
	goForward();
	%dopoki droga jest czysta to jedz
	while(lookDegrees(0)>1){
		continueWay();
	}
	%sciana, skret + zapisz dystans
	stop();
	distance = wheelCounter;
	turnRight();
	goForward();
	while(lookDegrees(0)>1){
		continueWay();
	}
	%okna, skret + reset counter
	stop();
	wheelCounter=0;
	turnRight();
	goForward();
	while(distance<wheelCounter){
		continueWay();
	}
	stop();
	turnRight();
	goForward();
	%witryna drzwi, zatrzymaj sie w progu
	while(lookRightDegrees(90)<70){
		continueWay();
	}
	stop();
	void increment() {
		if(digitalRead(INTINPUT0))
		wheelCounter++;
		}
}
\end{lstlisting}
\end{document}